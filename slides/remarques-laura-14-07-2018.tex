Slide 2 :
- La monotonie est-elle vraiment garantie (i.e. toutes les feuilles sont
  monotones) ?
  Quelle définition exactement ? Si A est l'arbre appris, F(A) la liste des
  étiquettes des feuilles, alors F(A) est croissante ou décroissante (évidemment
  c'est faux sur les exemples que tu m'as montrés) ?
- La dernière ligne est un peu vague à cause du "quand le domaine s'y prête". Tu
  pourrais changer ça en

    \textcite{pazzani-machin} :
    \begin{itemize}
      \item étude auprès d'experts : médecins
      \item gain d'interprétabilité
      \item performances équivalentes
    \end{itemize}

Slide 3 :
- Ici non plus tu n'es pas obligée de \beamcite, tu peux le mettre en \textcite.
- Peut-être mettre la colonne "prêt" en avant pour montrer que c'est la
  prédiction ?

Slide 4 :
Ça m'a l'air bien et lisible comme formalisation, au "monotone" près (cf.
remarque slide 2)

Slide 5 :
Donc la monotonie peut être fonction d'un seul attribut et pas nécessairement de
tous ? Par opposition à l'exemple suivant :

       ^
       |  x    o   a
       |    x   o     a
       |  x    o   a    a
     y |  x  x   o   a
       |    x  o   a      a
       |  x   x o  o  o o
       |  x  x   o  o  o  o
       |     x   x      x
       |     x    x   x
       -------------------->
               z
où x < o < a sont trois labels ordonnés tq.
    λ(ω₁) > λ(ω₂) ⇒ y(ω₁) > y(ω₂) ∨ z(ω₁) > z(ω₂)

ou à un exemple similaire en remplaçant le ∨ par un ∧

Slide 6 :
Ok ça répond à la question "monotonie globale". Du coup il faudrait idéalement
préciser l'idée derrière le "monotone" du début : c'est plutôt un "aussi monotone
que possible tout en maintenant des bonnes performances" ?

Slide 7 :
- Plutôt clair
- Un petit schéma pour le "construction hiérarchique" ? Juste une formule
  générique avec des couleurs + un exemple avec le même code couleur ?
  Ou alors tu en parles juste plus tard (mais il faut le dire)

Slide 8 :
- générées*
- Raisonnablement clair aussi, il faut juste bien les lire en mode "l'ensemble
  des points tels que" à l'oral
- pourrait bénéficier d'un schéma si tu trouves le temps mais ça n'est pas
  prioritaire

Slide 9 :
- Utilise $$ ou \begin{equation} plutôt que $
- Utilise \left( ... \right)   plutôt que ( ... ) pour un dimensionnement
  correct des parenthèses

Slide 10 :
- Le problème de citation semble venir du fait que ton .bib ne contient pas
  marsala-rank
- Cf. remarque slide 7
- Mesure de non-monotonie locale ? C'est-à-dire ?

Slide 15 :
Qui est qui ? Le deuxième est celui pour Shannon normale, i.e. sdm ?


Ça me semble être un très bon début. Comme dernier slide tu peux remettre une
frame avec le \maketitle, c'est généralement suffisant.
