\documentclass[a4paper]{article}

%% Language and font encodings
\usepackage[french]{babel}
\usepackage[utf8]{inputenc}
\usepackage[T1]{fontenc}

\usepackage{algorithm}
\usepackage[noend]{algpseudocode}
\algnewcommand{\algorithmicand}{\textbf{ and }}
\algnewcommand{\algorithmicor}{\textbf{ or }}
\algnewcommand{\OR}{\algorithmicor}
\algnewcommand{\AND}{\algorithmicand}
\algnewcommand\algorithmicforeach{\textbf{for each}}
\algdef{S}[FOR]{ForEach}[1]{\algorithmicforeach\ #1\ \algorithmicdo}


%% Sets page size and margins
\usepackage[a4paper,top=3cm,bottom=2cm,left=3cm,right=3cm,marginparwidth=1.75cm]{geometry}

%% Useful packages
\usepackage{amsmath}
\usepackage{amssymb}
\usepackage{graphicx}
\usepackage[colorinlistoftodos]{todonotes}
\usepackage[colorlinks=true, allcolors=blue]{hyperref}
\usepackage{graphicx}

\usepackage[backend=biber,uniquename=init,giveninits=true,
             %% "et al" pour > deux auteurs, & pour exactement 2
             uniquelist=false,maxcitenames=2,mincitenames=1,maxbibnames=99,
             isbn=false,url=false,doi=false,bibstyle=numeric
]{biblatex}
\addbibresource{references.bib}

\title{Document de travail}
\author{Laura Nguyen}

\begin{document}
\maketitle

\section{Introduction} Dans beaucoup de problèmes de classification, les valeurs
des attributs et de la classe sont ordinaux. De plus, il peut exister une
contrainte de monotonie: la classe d'un objet doit croître/décroître en fonction
de la valeur de tout ou partie de ses attributs.  A savoir, étant donné deux
objets $x, x'$, si $x \leq x'$ alors $f(x) \leq f(x')$. Les variables
dépendantes, $f(x)$ et $f(x')$, sont des fonctions monotones des variables
indépendantes, $x$ et $x'$.
On parle alors de problèmes de classification monotone, ou problèmes de
classification avec contrainte de monotonie. Cette contrainte indique que les
objets ayant de meilleures valeurs d'attributs ne doivent pas être assignés à de
moins bonnes valeurs de classe.\\
L'ajout de cette contrainte de monotonie permet d'introduire des concepts
sémantiques tels la préférence, la priorité, l'importance, qui nécessitent une
relation d'ordre.\\ Il existe de nombreux domaines se prêtant à ce type de
tâches, tels la prédiction du risque de faillite \cite{greco-new-bankruptcy},
l'analyse de la satisfaction des clients \cite{greco-customer}, le diagnostic
médical \cite{marsala-gradual}. 
L'importance de la prise en compte d'une relation graduelle entre les valeurs
d'attributs et la classe a été démontrée \cite{pazzani-acceptance}: les
classifieurs auxquels sont imposés la contrainte de monotonie sont au moins
aussi performants que leurs homologues classiques, et les experts sont plus
enclins à utiliser les règles générées par les modèles monotones.\\
Afin d'extraire des règles à partir de données monotones, on décide d'utiliser
les arbres de décision, dont l'efficacité et l'interprétabilité en
classification a été prouvée \cite{quinlan-induction}.  Cependant, les
algorithmes de construction d'arbres de décision standards (générés par CART
\cite{leo-classification}) ne produisent pas de classifieurs sensibles à la
monotonie, même si la base utilisée est complètement monotone.  En revanche, il
est montré dans \cite{ben-adding} que les classifieurs purement monotones
(\cite{ben-learning}, \cite{ben-monotonicity}, \cite{cao-consistent}) sont, en
terme de taux de bonne classification, statistiquement indiscernables de leurs
homologues non-monotones.  Dans le même article, il est expliqué que ce
phénomène est dû à la sensibilité de ces classifieurs au bruit non-monotone
présent dans les données réelles. \\

Ce stage a pour but d'étudier la construction et l'évaluation d'arbres de
décision prenant en compte une relation graduelle susceptible d'exister entre
les valeurs d'attributs et la classe, tout en étant suffisamment robuste au
bruit non-monotone. On reprend, en particulier, \cite{marsala-rank} pour la
construction d'arbres de décision monotones paramétrés par une mesure de
discrimination à rang. Une étude théorique des propriétés des mesures présentées
dans le même article est également effectuée.\\

%\section{Etat de l'art}

\section{Implémentation et expérimentation de l'algorithme de construction
d'arbres monotones} 
Dans cette partie, on implémente (à quelques variantes près)
RDMT(H), l'algorithme de construction d'arbres monotones donné dans
\cite{marsala-rank} et on l'évalue sur des données artificielles et réelles. 

\subsection{Implémentation des mesures de discrimination à rang} 
D'après
\cite{marsala-rank}, les mesures de discrimination à rang possèdent la même
structure fonctionnelle: elles se décomposent en trois fonctions.  Dans le même
article, un modèle de construction hiérarchique de mesures de discrimination à
rang est proposé. Il permet d'isoler leurs propriétés et d'en créer de
nouvelles. 



\printbibliography
\end{document}
